\input{preambuloSimple.tex}

%----------------------------------------------------------------------------------------
%	TÍTULO Y DATOS DEL ALUMNO
%----------------------------------------------------------------------------------------

\title{	
\normalfont \normalsize 
\textsc{\textbf{Ingeniería de Servidores (2021-2022)} \\ Grado en Ingeniería Informática \\ Universidad de Granada} \\ [25pt] % Your university, school and/or department name(s)
\horrule{0.5pt} \\[0.4cm] % Thin top horizontal rule
\huge Memoria Práctica 3 \\ % The assignment title
\horrule{2pt} \\[0.5cm] % Thick bottom horizontal rule
}

\author{Cristina Sánchez Justicia} % Nombre y apellidos

\date{\normalsize\today} % Incluye la fecha actual

%----------------------------------------------------------------------------------------
% DOCUMENTO
%----------------------------------------------------------------------------------------

\begin{document}

\maketitle % Muestra el Título

\newpage %inserta un salto de página

\tableofcontents % para generar el índice de contenidos

\listoffigures

\listoftables

\newpage

%------------------------------------------------------------------
%	Pregunta 1
%------------------------------------------------------------------

\section{Realice una instalación de Zabbix 5.0 en su servidor con Ubuntu Server20.04 y configure para que se monitorice a él mismo y para que monitorice a la máquina con CentOS. 
Puede configurar varios parámetros para monitorizar, uso de CPU, memoria, etc, pero debe configurar de manera obligatoria la monitorización de los servicios SSH y HTTP. 
Documente el proceso de instalación y configuración indicando las referencias que ha utilizado así como los problemas que ha encontrado. Para ello, se le recomienda utilizar la plantilla de \LaTeX disponible en SWAD. Procure que en las capturas aparezca su nombre de usuario (en el prompt p.ej. como hemos hecho en los exámenes). El archivo debe estar subido a SWAD (zonamis trabajos) antes del final del curso (ver actividad en SWAD)}

\subsection{Instalación}
El primer paso es arrancar en VirtualBox mi Ubuntu Server20.04. \newline 
Mi primera fuente de información ha sido la página recomendada en las prácticas % https://www.zabbix.com/documentation/5.0/en/manual 
\newline
En ella me llama la atención decargar zabbox desde la distribuición pre-compilada de la página web de Zabbix pero desde la máquina virtual no tengo esa opción así que deberé utilizar otra. 
\newline
Como segunda opción me baso en la documentación oficial. % https://www.zabbix.com/download?zabbix=5.0&os_distribution=ubuntu&os_version=20.04_focal&db=mysql&ws=apache donde sí hay instrucciones para instalar en un servidor. 
\newline
Escribo el sigiente comando en ubuntu pero me da error: 
wget % https://repo.zabbix.com/zabbix/5.0/ubuntu/pool/main/z/zabbix-release/zabbix-release_5.0-1+focal_all.deb
\newline
Me da un error:
\begin{figure}[H]
\centering
\includegraphics[scale=0.5]{Error1.jpg}
\end{figure} 

\subsection{Monitorización}
\subsubsection{A él mismo}
SSH y HTTP
\subsubsection{Máquina con CentOS}
\newpage
\end{document}
