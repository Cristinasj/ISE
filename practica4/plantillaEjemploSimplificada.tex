\input{preambuloSimple.tex}

%----------------------------------------------------------------------------------------
%	TÍTULO Y DATOS DEL ALUMNO
%----------------------------------------------------------------------------------------

\title{	
\normalfont \normalsize 
\textsc{\textbf{Ingeniería de Servidores (2021-2022)} \\ Grado en Ingeniería Informática \\ Universidad de Granada} \\ [25pt] % Your university, school and/or department name(s)
\horrule{0.5pt} \\[0.4cm] % Thin top horizontal rule
\huge Memoria Práctica 4 \\ % The assignment title
\horrule{2pt} \\[0.5cm] % Thick bottom horizontal rule
}

\author{Cristina Sánchez Justicia} % Nombre y apellidos

\date{\normalsize\today} % Incluye la fecha actual

%----------------------------------------------------------------------------------------
% DOCUMENTO
%----------------------------------------------------------------------------------------

\begin{document}

\maketitle % Muestra el Título

\newpage %inserta un salto de página

\tableofcontents % para generar el índice de contenidos

\listoffigures

\listoftables

%----------------------------------------------------------------------------------------
%	Cuestión 1
%------------------------------------------------https://linuxconfig.org/how-to-install-the-latest-phoronix-test-suite-on-ubuntu-18-04-bionic-beaver----------------------------------------

\section{Una vez que haya indagado sobre los benchmarks disponibles, seleccione como mínimo dos de ellos y proceda a ejecutarlos en Ubuntu y CentOS.
Comente las diferencias}

Instalo phronix en Ubuntu siguiendo este tutorial "https://linuxconfig.org/how-to-install-the-latest-phoronix-test-suite-on-ubuntu-18-04-bionic-beaver"
%\b{sudo apt install gdebi-core}
%\b{sudo gdebo phoronix-test-suite_7.8.0_all.deb}
Sin embargo, me da error 
\begin{figure}[H]
\centering
\includegraphics{phroroerror.jpg}
\end{figure} 
Pruebo otra vez con la otra forma que dicen y esta vez funciona 
\begin{figure}[H]
\centering
\includegraphics{pbien.jpg}
\end{figure} 
Me doy cuenta de que lo que he hecho antes era el segundo paso y no otro método a parte. Vuelvo a repetirlo para instalar el paquete. 
\begin{figure}[H]
\centering
\includegraphics{gdebi.jpg}
\end{figure} 
Ejecuto el programa con phoronix-test-suite y pregunto por los benchmarks disponibles con phoronix-test-suite list-available-tests.  
\begin{figure}[H]
\centering
\includegraphics{availabletests.jpg}
\end{figure} 
Ahora tengo que elegir 2 tests y ejecutarlos. Voy a elegir los 2 primeros: pts/ai-benchmark y pts/aircrack-ng. 
Para el primero me ha avisado de que me faltan dependencias y se ha puesto a instalarlas: 
\begin{figure}[H]
\centering
\includegraphics{deviam.jpg}
\end{figure}
Resultado: 
\begin{figure}[H]
\centering
\includegraphics{resultado.jpg}
\end{figure}
Para la segunda benckmark me vuelve a decir que tengo que instalarlo y que para ello me faltan dependencias. Le vuelvo a pedir que lo haga, pero esta vez no consigue resolver las dependencias así que escojo en su lugar el tercer benckmark: pts/amg 
Repito todo el proceso de antes, esta vez exitosamente. Lo vuelvo a ejecutar y este es el resultado: 
\begin{figure}[H]
\centering
\includegraphics{resultado3.jpg}
\end{figure}
Es igual que el resultado anterior. 

\section{tras probar un test básico para una web, utilizaremos Jmeter para hacer un test sobre una aplicación que ejecuta sobre dos contenedores (uno para la BD y otro para la aplicación en sí). El código está disponible en https://github.com/davidPalomar-ugr/iseP4JMeter donde se dan detalles sobre cómo
ejecutar la aplicación en una de nuestras máquinas virtuales. El test de Jmeter debe incluir los siguientes elementos:
El test debe tener parametrizados el Host y el Puerto en el Test Plan (puede hacer referencia usando param)
Debe hacer dos grupos de hebras distintos para simular el acceso de los alumnos y los administradores. Las credenciales de alumno y administrador se cogen de los archivos: alumnos.csv y administrador.csv respectivamente.
Añadimos esperas aleatorias a cada grupo de hebras (Gaussian Random
Timer)
El login de alumno, su consulta de datos (recuperar datos alumno) y login del adminsitrador son peticiones HTTP.
El muestreo para simular el acceso de los administradores lo debe coger el archivo apiAlumnos.log (usando un Acces Log Sampler)
Use una expresión regular (Regular Expressión Extractor) para extraer el token JWT que hay que añadir a la cabecera de las peticiones (usando HTTP Header Manager)}
En primer lugar instalo JMeter en Ubuntu. Para eso voy a necesitar docker y docker-compose, que instalo con apt. 
Después clono el repositorio del profesor con git clone. 
Hago sudo docker-compose up pero me dice que el servidor no está corriendo
\begin{figure}[H]
\centering
\includegraphics{runin.jpg}
\end{figure}
Le hago sudo systemclt start docker y ya puedo hacerle docker compose up. 
Este es el resultado: 
\begin{figure}[H]
\centering
\includegraphics{rebuild.jpg}
\end{figure}
Hago sudo docker-compose build como me pide. Me sigue dando el error del espacio en disco. 

\end{document}
