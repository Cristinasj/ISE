\input{preambuloSimple.tex}

%----------------------------------------------------------------------------------------
%	TÍTULO Y DATOS DEL ALUMNO
%----------------------------------------------------------------------------------------

\title{	
\normalfont \normalsize 
\textsc{\textbf{Ingeniería de Servidores (2021-2022)} \\ Grado en Ingeniería Informática \\ Universidad de Granada} \\ [25pt] % Your university, school and/or department name(s)
\horrule{0.5pt} \\[0.4cm] % Thin top horizontal rule
\huge Memoria Práctica 4 \\ % The assignment title
\horrule{2pt} \\[0.5cm] % Thick bottom horizontal rule
}

\author{Cristina Sánchez Justicia} % Nombre y apellidos

\date{\normalsize\today} % Incluye la fecha actual

%----------------------------------------------------------------------------------------
% DOCUMENTO
%----------------------------------------------------------------------------------------

\begin{document}

\maketitle % Muestra el Título

\newpage %inserta un salto de página

\tableofcontents % para generar el índice de contenidos

\listoffigures

\listoftables

%----------------------------------------------------------------------------------------
%	Cuestión 1
%------------------------------------------------https://linuxconfig.org/how-to-install-the-latest-phoronix-test-suite-on-ubuntu-18-04-bionic-beaver----------------------------------------

\section{Una vez que haya indagado sobre los benchmarks disponibles, seleccione como mínimo dos de ellos y proceda a ejecutarlos en Ubuntu y CentOS.
Comente las diferencias}

Instalo phronix en Ubuntu siguiendo este tutorial "https://linuxconfig.org/how-to-install-the-latest-phoronix-test-suite-on-ubuntu-18-04-bionic-beaver"
%\b{sudo apt install gdebi-core}
%\b{sudo gdebo phoronix-test-suite_7.8.0_all.deb}
Sin embargo, me da error 
\begin{figure}[H]
\centering
\includegraphics{phroroerror.jpg}
\end{figure} 
Pruebo otra vez con la otra forma que dicen y esta vez funciona 
\begin{figure}[H]
\centering
\includegraphics{pbien.jpg}
\end{figure} 
Me doy cuenta de que lo que he hecho antes era el segundo paso y no otro método a parte. Vuelvo a repetirlo para instalar el paquete. 
\begin{figure}[H]
\centering
\includegraphics{gdebi.jpg}
\end{figure} 
Ejecuto el programa con phoronix-test-suite y pregunto por los benchmarks disponibles con phoronix-test-suite list-available-tests.  
\begin{figure}[H]
\centering
\includegraphics{availabletests.jpg}
\end{figure} 

\subsection{Benchmars disponibles}

\end{document}
